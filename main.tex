\documentclass[a4paper,12pt]{article}
\usepackage{CommandsAndStyle}
% \usepackage{forloop}

\newcounter{ProblemNumber}

\newcommand\Problem[2]{
	\addtocounter{ProblemNumber}{1}%
	\expandafter\def\csname Problem_\theProblemNumber\endcsname{#1}%
	\subsection{#1}\label{Problem_\theProblemNumber}#2%
}

\newcommand\Hint[1]{
	\expandafter\def\csname Hint_\theProblemNumber\endcsname{#1}%
	\paragraph{Hint:} Clicca \hyperref[Hint_\theProblemNumber]{qui} per un hint.%
}

\newcommand\Solution[1]{\paragraph{Dimostrazione}\begin{proof}#1\end{proof}}
\newcommand\Comment[1]{\paragraph{Commento:}#1}

\title{Raccolta di problemi matematici}
\author{Federico Glaudo}

% \makeindex[title=Indice analitico]
% \indexsetup{level=\section}

\begin{document}

\maketitle

\begin{abstract}
	Dopo un anno di matematica universitaria mi sono accorto della rarità e del valore dei bei problemi. Sono pochi quelli adatti ad entrare in questa categoria: devono essere ``fattibili''\footnote{Cosa c'è di più soggettivo della difficoltà? Qui si intende che la soluzione di ogni problema è, con i dovuti strumenti, non più di un'idea ben scelta.} e piacevoli tanto nell'enunciato quanto nella soluzione.
	
	L'argomento dei problemi sarà vario, ma la difficoltà non poi molto. Alcuni dei problemi sono celeberrimi, altri misconosciuti. Alcuni si prestano anche a soluzioni brutte, ma tutti sono risolvibili elegantemente con un guizzo d'ingegno.
	
	Questo documento è pensato per essere \emph{consultato in versione digitale} piuttosto che cartacea. 
	Infatti i testi dei problemi sono tutti all'inizio ed ognuno linka agli eventuali hint e soluzioni relativi ad esso. Sia gli hint che le soluzioni sono \emph{uno per pagina}, affinché non capiti, per sbaglio o per la debolezza di un momento, di leggere un hint o una soluzione a cui non si è interessati. Ovviamente anche gli hint e le soluzioni espongono un link che rimanda al testo del problema. 
	%TODO: Completare qui, dicendo che strumenti servono e perché risolverli.
\end{abstract}
\clearpage

\tableofcontents
\clearpage

\section{Algebra}

\Problem{I gruppi con ordine divisibile esattamente per \texorpdfstring{$2$}{2} non sono semplici}{
Dimostrare che se $G$ è un gruppo con ordine $\abs{G}=2d$ dove $d$ è un intero dispari diverso da $1$, allora $G$ non è semplice.
}
\Hint{Ricordate la dimostrazione del teorema di Cayley?}

\Problem{Anello PID non Euclideo}{
Chiamando $\xi$ una radice del polinomio $x^2+x+5$, dimostrare che l'anello $\Z[\xi]$ è ad ideali principali ma non ammette una divisione Euclidea.
}

\Problem{Somma di radici quadrate}{
Dimostrare che $\alpha=\sqrt2+\sqrt{6}+\sqrt{10}+\sqrt{15}+\sqrt{37}$ è algebrico su $\Q$ e che il suo polinomio minimo ha grado $16$.
}
\Hint{Estendete $\Q$ con le radici quadrate dei numeri $2,3,5,37$. Che grado ha l'estensione? Quale è il suo gruppo di Galois?}

\Problem{Serie nulla \texorpdfstring{in $\Q_2$}{nei numeri 2-adici}}{
Dimostrare che $\sum_{n=1}^\infty \frac{2^n}{n}$ converge a $0$ in $\Q_2$\footnote{Con $\Q_2$ si intende la chiusura topologica dei razionali rispetto al valore assoluto $2$-adico. Usando termini più elementari, il problema chiede di dimostrare che le troncate della serie hanno molti fattori $2$.}.
}
\Hint{Ricordate lo sviluppo del logaritmo?}
\section{Analisi}

\Problem{Contrazioni in un compatto}{
Sia $(X,d)$ uno spazio metrico compatto ed $f:X\to X$ una funzione tale che
\begin{equation*}
	\forall x,y\in X: d\left(f(x),f(y)\right)<d(x,y)\punto
\end{equation*}
Allora la funzione $f$ ammette un punto fisso.
}
\Hint{La funzione $x\mapsto d(x,f(x))$ ammette minimo.}

\Problem{Punti di discontinuità}{
Esiste una funzione $f:\R\to\R$ tale che sia continua in $x\in\R$ se e solo se $x\in\Q$?
}
\Hint{La risposta è no e per dimostrarlo si può passare dal fatto che i punti di discontinuità di $f$ sono un insieme magro (nell'ipotesi in cui su $\Q$ la funzione sia continua) e concludere con Baire.}

\Problem{Misura del tetraedro \texorpdfstring{$n$}{n}-dimensionale}{
Calcolare la misura del tetraedro $n$-dimensionale di lato unitario. Con tetraedro $n$-dimensionale si intende l'inviluppo convesso di $n+1$ punti in $\R^n$ equidistanti tra di loro.
}
\Hint{E se calcolassimo la misura in $\R^{n+1}$ al poso che in $\R^n$? Allora potremmo scegliere come vertici del tetraedro la base canonica...}

\Problem{Integrale di Eulero}{
Dimostrare l'uguaglianza
\begin{equation*}
	\int_{-\infty}^{\infty}e^{-x^2}\de x=\sqrt\pi
\end{equation*}
}
\Hint{Considerate l'identità
\begin{equation*}
	\left(\int_{-\infty}^{\infty}e^{-x^2}\right)^2 = \int_{\R^2}e^{-(x^2+y^2)}\de(x,y)\punto
\end{equation*}
Alternativamente, se conoscete la funzione gamma di Eulero, potreste sfruttare il fatto che, ad esempio con la formula di duplicazione, potete calcolare $\Gamma(\frac12)$.
}
\section{Geometria}

\Problem{Spazio euclideo senza alcuni punti}{
Fissato $n\ge 1$ e $h\not=k$ interi non negativi distinti, dimostrare che $\R^n$ senza $h$ punti non è omeomorfo a $\R^n$ senza $k$ punti.
}
\Hint{Considerate un'esaustione in compatti.}

\Problem{Gruppo fondamentale di una sfera con punti identificati}{
Sia $X$ lo spazio ottenuto identificando due punti di $S^2$. Chi è il gruppo fondamentale di $X$?
}

\Problem{Segmenti paralleli su una curva}{
Sia $\gamma:\cc{0}{1}\to\R^2$ una curva continua con $\gamma(0)=(0, 0)$ e $\gamma(1)=(1, 0)$.
Mostrare che, fissato $0<d<1$, esistono due tempi $t_1,t_2\in\cc{0}{1}$ tali che $\gamma(t_1)$ e $\gamma(t_2)$ formano un segmento parallelo all'asse delle $x$ e di lunghezza $d$ o $1-d$.
}
\Hint{È equivalente ad affermare che una curva sul cilindro omotopica al generatore del gruppo fondamentale interseca una rotazione di se stessa. A questo punto, dopo aver ``pulito'' $\gamma$, si ritorna opportunamente sul piano e si nota che qualcosa non torna per Jordan.}

\Problem{Coni convessi chiusi}{
	Dato un cono convesso chiuso $C\subseteq \R^n$ sono equivalenti le seguenti affermazioni:
	\begin{itemize}
		\item Una retta è contenuta in $C$.
		\item Per ogni funzionale $L\in(\R^n)'$ esiste $x\in C\setminus\{0\}$ tale che $Lx\ge 0$.
	\end{itemize}
}
\Hint{Considerate $C\cap S^{n-1}$.}
\Solution{Mostriamo che se $C$ non contiene rette allora esiste un funzionale che valutato su $C$ è ovunque negativo.
	
	L'insieme $D = C\cap S^{n-1}$ è chiuso e convesso e senza punti antipodali.
	
	Intersecando $D$ con $x_n=0$, per induzione, si trova che, senza perdita di generalità, $D$ è disgiunto dall'insieme $x_{n-1}\le 0$.
	
	Denotiamo ora con $S$ il sottospazio vettoriale di $\R^n$ dato da $x_n=x_{n-1}=0$ e con $P$ il piano ortogonale ad $S$. Definiamo ora la funzione angolo $\theta:P\to S^1$ che associa ad ogni elemento di $P$ l'angolo che forma con il vettore $e_n$. È ovvio che, a meno di proiettare su $P$, la funzione $\theta$ si estende a tutto $S^{n-1}$. % FIXME: Sarebbe da definire meglio l'angolo
	
	Chiamiamo $A$ l'immagine di $D$ tramite $\theta$. 
	Per quanto detto sopra, di certo $A$ non contiene $0$. Inoltre, visto che $D$ è convesso, anche $A$ è convesso.
	Allora, facilmente, si ha che $A$ è contenuto in una semicirconferenza e, prendendo la controimmagine della semicirconferenza tramite $\theta$, si trova che $D$ è contenuto in un semispazio.
	
	Rimane solo da verificare che, a meno di spostare di pochissimo questo semispazio, $D$ non è neanche sul bordo del semispazio e quindi esiste un funzionale che falsifica l'ipotesi. Lasciamo al lettore quest'ultimo passo.
}

\Problem{Allacciando colonne}{
	Re Mida ha un salone rotondo con $n\in\N$ colonne ed ha una corda d'oro di inestimabile valore (che potete assumere lunga a piacere). La corda è ``aperta'', nel senso che le estremità non sono collegate.
	
	Egli vuole allacciare (cioè disporre a piacimento e poi fondere le estremità) la corda intorno alle colonne in modo che non sia possibile rubarla. 
	Teme però il cedimento del salone e vorrebbe perciò che fosse possibile portare via la corda non appena una qualunque delle $n$ colonne crolli.
	Può re Mida riuscire in questo particolare allacciamento?
}
\Hint{Considerate i gruppi fondamentali (e le loro immersioni canoniche l'uno nell'altro) del piano euclideo senza $n$ o $n-1$ punti. Mostrate quindi per induzione che Re Mida può allacciare la corda come richiesto. }
\Solution{Denotiamo con $\pi(p_1, p_2, \dots, p_k)$ il gruppo fondamentale di $\R^2$ privato dei punti $p_1, p_2,\dots, p_k$ (tali punti rappresentano le colonne del salone). 
	
	È fatto noto che come gruppo $\pi(p_1, p_2, \dots p_k)$ è $\overbrace{\Z\ast \cdots \ast \Z}^{\text{$k$ volte}}$.
	Inoltre l'omomorfismo canonico indotto dall'immersione da $\pi(p_1, p_2, \dots p_k)$ in $\pi(p_1, p_2, \dots, \hat{p_i}, \dots, p_k)$ visto sui rappresentanti canonici dei gruppi liberi consiste nell'ignorare l'$i$-esimo rappresentante. Denotiamo con $\iota_i^k$ la mappa appena descritta tra i gruppi liberi
	\begin{equation}
		\iota_i^k:\overbrace{\Z\ast \cdots \ast \Z}^{\text{$k$ volte}} \to \overbrace{\Z\ast \cdots \ast \Z}^{\text{$k-1$ volte}} \punto
	\end{equation}
	Il problema è ora equivalente a chiedersi se esiste una parola $w\in\overbrace{\Z\ast \cdots \ast \Z}^{\text{$n$ volte}}$ diversa dall'identità tale che $\iota_i^n(w)$ è l'identità per ogni $1\le i\le n$.
	
	Mostriamo per induzione su $n$ che tale parola esiste. Per $n=1$ è banale. Se $w_{n-1}$ è una parola che rispetta per $n-1$, è facile convincersi che $x_nw_{n-1}x_n^{-1}w_{n-1}^{-1}$ funziona per $n$ (dove con $x_1,\dots, x_n$ stiamo denotando i rappresentanti canonici dei gruppi liberi).
}
%
\addtocounter{ProblemNumber}{1}%
\newcounter{ActProb}%
\forloop{ActProb}{1}{\number\value{ActProb} < \number\value{ProblemNumber}}{%
	\ifcsname Hint_\theActProb\endcsname%
		\paragraph{Hint per ``\protect\csname Problem_\theActProb\endcsname'':}\label{Hint_\theActProb}%
		\csname Hint_\theActProb\endcsname \ (\hyperref[Problem_\theActProb]{Testo del problema})%
	\fi%
}%
%
\end{document}