\documentclass[a4paper,12pt]{article}
\usepackage{CommandsAndStyle}
% \usepackage{forloop}

\newcounter{ProblemNumber}

\newcommand\Problem[2]{
	\addtocounter{ProblemNumber}{1}%
	\expandafter\def\csname Problem_\theProblemNumber\endcsname{#1}%
	\subsection{#1}\label{Problem_\theProblemNumber}#2%
}

\newcommand\Hint[1]{
	\expandafter\def\csname Hint_\theProblemNumber\endcsname{#1}%
	\paragraph{Hint:} Clicca \hyperref[Hint_\theProblemNumber]{qui} per un hint.%
}

\newcommand\Solution[1]{\paragraph{Dimostrazione}\begin{proof}#1\end{proof}}
\newcommand\Comment[1]{\paragraph{Commento:}#1}

\title{Raccolta di problemi matematici}
\author{Federico Glaudo}

\makeindex[title=Indice analitico]
\indexsetup{level=\section}

\begin{document}

\maketitle

\begin{abstract}
	Dopo un anno di matematica universitaria mi sono accorto della rarità e del valore dei bei problemi. Sono pochi quelli adatti ad entrare in questa categoria: devono essere ``fattibili''\footnote{Cosa c'è di più soggettivo della difficoltà? Qui si intende che la soluzione di ogni problema è, con i dovuti strumenti, non più di un'idea ben scelta.} e piacevoli tanto nell'enunciato quanto nella soluzione.
	
	L'argomento dei problemi sarà vario, ma la difficoltà non poi molto. Alcuni dei problemi sono celeberrimi, altri misconosciuti. Alcuni si prestano anche a soluzioni brutte, ma tutti sono risolvibili elegantemente con un guizzo d'ingegno.
	
	Questo documento è pensato per essere \emph{consultato in versione digitale} piuttosto che cartacea. 
	Infatti i testi dei problemi sono tutti all'inizio ed ognuno linka agli eventuali hint e soluzioni relativi ad esso. Sia gli hint che le soluzioni sono \emph{uno per pagina}, affinché non capiti, per sbaglio o per la debolezza di un momento, di leggere un hint o una soluzione a cui non si è interessati. Ovviamente anche gli hint e le soluzioni espongono un link che rimanda al testo del problema. 
	%TODO: Completare qui, dicendo che strumenti servono e perché risolverli.
\end{abstract}
\clearpage

\tableofcontents
\clearpage

\section{Algebra}

\Problem{I gruppi con ordine divisibile esattamente per \texorpdfstring{$2$}{2} non sono semplici}{
Dimostrare che se $G$ è un gruppo con ordine $\abs{G}=2d$ dove $d$ è un intero dispari diverso da $1$, allora $G$ non è semplice.
}
\Hint{Ricordate la dimostrazione del teorema di Cayley?}

\Problem{Anello PID non Euclideo}{
Chiamando $\xi$ una radice del polinomio $x^2+x+5$, dimostrare che l'anello $\Z[\xi]$ è ad ideali principali ma non ammette una divisione Euclidea.
}

\Problem{Somma di radici quadrate}{
Dimostrare che $\alpha=\sqrt2+\sqrt{6}+\sqrt{10}+\sqrt{15}+\sqrt{37}$ è algebrico su $\Q$ e che il suo polinomio minimo ha grado $16$.
}
\Hint{Estendete $\Q$ con le radici quadrate dei numeri $2,3,5,37$. Che grado ha l'estensione? Quale è il suo gruppo di Galois?}

\Problem{Serie nulla \texorpdfstring{in $\Q_2$}{nei numeri 2-adici}}{
Dimostrare che $\sum_{n=1}^\infty \frac{2^n}{n}$ converge a $0$ in $\Q_2$\footnote{Con $\Q_2$ si intende la chiusura topologica dei razionali rispetto al valore assoluto $2$-adico. Usando termini più elementari, il problema chiede di dimostrare che le troncate della serie hanno molti fattori $2$.}.
}
\Hint{Ricordate lo sviluppo del logaritmo?}

\Problem{Potenza simmetrica implica quadrato simmetrico}{
Sia $A$ un dominio di integrità ed $f\in A[x_1,\dots,x_k]$ un polinomio in $k$ variabili.
Se $f^n$ è un polinomio simmetrico per qualche $n\in\N$, allora $f^2$ è un polinomio simmetrico.
}
\Hint{Estendere $A$ ad una chiusura algebrica, per poter fattorizzare $f^n-(f\circ\tau)^n$ dove $\tau$ è una traspozione.}

\Problem{Potenza intera implica esponente intero}{
Sia $x\in\R$ tale che, per ogni primo $p$ sufficientemente grande, sia ha $p^x$ intero.
Mostrare che $x$ è un intero positivo.
}
\Hint{Quanto vale la derivata di $t^x$?}
\section{Analisi}

\Problem{Contrazioni in un compatto}{
Sia $(X,d)$ uno spazio metrico compatto ed $f:X\to X$ una funzione tale che
\begin{equation*}
	\forall x,y\in X: d\left(f(x),f(y)\right)<d(x,y)\punto
\end{equation*}
Allora la funzione $f$ ammette un punto fisso.
}
\Hint{La funzione $x\mapsto d(x,f(x))$ ammette minimo.}

\Problem{Punti di discontinuità}{
Esiste una funzione $f:\R\to\R$ tale che sia continua in $x\in\R$ se e solo se $x\in\Q$?
}
\Hint{La risposta è no e per dimostrarlo si può passare dal fatto che i punti di discontinuità di $f$ sono un insieme magro (nell'ipotesi in cui su $\Q$ la funzione sia continua) e concludere con Baire.}

\Problem{Misura del tetraedro \texorpdfstring{$n$}{n}-dimensionale}{
Calcolare la misura del tetraedro $n$-dimensionale di lato unitario. Con tetraedro $n$-dimensionale si intende l'inviluppo convesso di $n+1$ punti in $\R^n$ equidistanti tra di loro.
}
\Hint{E se calcolassimo la misura in $\R^{n+1}$ al poso che in $\R^n$? Allora potremmo scegliere come vertici del tetraedro la base canonica...}

\Problem{Integrale di Eulero}{
Dimostrare l'uguaglianza
\begin{equation*}
	\int_{-\infty}^{\infty}e^{-x^2}\de x=\sqrt\pi
\end{equation*}
}
\Hint{Considerate l'identità
\begin{equation*}
	\left(\int_{-\infty}^{\infty}e^{-x^2}\right)^2 = \int_{\R^2}e^{-(x^2+y^2)}\de(x,y)\punto
\end{equation*}
Alternativamente, se conoscete la funzione gamma di Eulero, potreste sfruttare il fatto che, ad esempio con la formula di duplicazione, potete calcolare $\Gamma(\frac12)$.
}

\Problem{Quasi l'esponenziale}{
Fissato $k=10^{10000}$, mostrate la stima
\begin{equation*}
	\left\lvert\sum_{n=0}^k (-1)^n \frac{k^n}{n!^2}\right\lvert\le 1 \punto
\end{equation*}
Riuscite a migliorarla?
}
\Hint{Il comportamento asintotico delle soluzioni di equazioni differenziali può essere studiato con stime di tipo energia.}
\Solution{Sia $y:\R\to\R$ la funzione analitica definita come
\begin{equation*}
	y(x)=\sum_{n=0}^{\infty} (-1)^n\frac{x^n}{n!^2}
\end{equation*}
e sia $u:\R\to\R$ definita come $u(x)=x\cdot y'(x)$. 
È facile verificare che $u''=-\frac ux$ e che quindi la quantità $E(x)=\frac{u^2}x+{u'}^2$ ha come derivata $-\frac{u^2}{x^2}$.

A questo punto notiamo, che per $x\ge 1$, vale
\begin{equation*}
	y(x)^2=u'(x)^2\le E(x)\le E(1)=u(1)^2+{u'(1)}^2\le \left(\frac14\right)^2+\left(\frac12\right)^2<\frac12 \virgola
\end{equation*}
da cui la tesi discende facilmente.

Per avere stime migliore per $y$, si deve studiare come si comporta asintoticamente l'energia. Euristicamente sembra valere $E(x)\propto\frac1{\log^2(x)}$.
}

\Problem{Differenziale antisimmetrico}{
Se $f:\R^n\to\R^n$ è una funzione tale che il differenziale $\de f$ è antisimmetrico in ogni punto, allora $f$ è una funzione affine.
}
\Hint{Applicate il lemma di Schwartz (la commutazione delle derivate) alle componenti di $f$.}

\Problem{Serie alternante}{
Dati due numeri reali positivi $a,b$ con $b<1$, la serie $\sum_{n=1}^\infty \frac{\cos n^b}{n^a}$ converge se e solo se $a+b>1$.
}
\Hint{Quando $a+b\le 1$, al variare di $k\in\N$, considerate la somma parziale
\begin{equation*}
	\sum_{2k\pi\le n^b\le (2k+\frac14)\pi}\frac{\cos n^b}{n^a}\punto
\end{equation*}
Viceversa, per mostrare la convergenza, provate a riscrivere la serie come somma di Abel.
}
\section{Geometria}

\Problem{Spazio euclideo senza alcuni punti}{
Fissato $n\ge 1$ e $h\not=k$ interi non negativi distinti, dimostrare che $\R^n$ senza $h$ punti non è omeomorfo a $\R^n$ senza $k$ punti.
}
\Hint{Considerate un'esaustione in compatti.}

\Problem{Gruppo fondamentale di una sfera con punti identificati}{
Sia $X$ lo spazio ottenuto identificando due punti di $S^2$. Chi è il gruppo fondamentale di $X$?
}
%
\addtocounter{ProblemNumber}{1}%
\newcounter{ActProb}%
\forloop{ActProb}{1}{\number\value{ActProb} < \number\value{ProblemNumber}}{%
	\ifcsname Hint_\theActProb\endcsname%
		\paragraph{Hint per ``\protect\csname Problem_\theActProb\endcsname'':}\label{Hint_\theActProb}%
		\csname Hint_\theActProb\endcsname \ (\hyperref[Problem_\theActProb]{Testo del problema})%
	\fi%
}%
%
\end{document}