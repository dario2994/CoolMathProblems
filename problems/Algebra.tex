\section{Algebra}

\Problem{I gruppi con ordine divisibile esattamente per \texorpdfstring{$2$}{2} non sono semplici}{
Dimostrare che se $G$ è un gruppo con ordine $\abs{G}=2d$ dove $d$ è un intero dispari diverso da $1$, allora $G$ non è semplice.
}
\Hint{Ricordate la dimostrazione del teorema di Cayley?}

\Problem{Anello PID non Euclideo}{
Chiamando $\xi$ una radice del polinomio $x^2+x+5$, dimostrare che l'anello $\Z[\xi]$ è ad ideali principali ma non ammette una divisione Euclidea.
}

\Problem{Somma di radici quadrate}{
Dimostrare che $\alpha=\sqrt2+\sqrt{6}+\sqrt{10}+\sqrt{15}+\sqrt{37}$ è algebrico su $\Q$ e che il suo polinomio minimo ha grado $16$.
}
\Hint{Estendete $\Q$ con le radici quadrate dei numeri $2,3,5,37$. Che grado ha l'estensione? Quale è il suo gruppo di Galois?}

\Problem{Serie nulla \texorpdfstring{in $\Q_2$}{nei numeri 2-adici}}{
Dimostrare che $\sum_{n=1}^\infty \frac{2^n}{n}$ converge a $0$ in $\Q_2$\footnote{Con $\Q_2$ si intende la chiusura topologica dei razionali rispetto al valore assoluto $2$-adico. Usando termini più elementari, il problema chiede di dimostrare che le troncate della serie hanno molti fattori $2$.}.
}
\Hint{Ricordate lo sviluppo del logaritmo?}