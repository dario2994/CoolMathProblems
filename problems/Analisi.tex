\section{Analisi}

\Problem{Contrazioni in un compatto}{
Sia $(X,d)$ uno spazio metrico compatto ed $f:X\to X$ una funzione tale che
\begin{equation*}
	\forall x,y\in X: d\left(f(x),f(y)\right)<d(x,y)\punto
\end{equation*}
Allora la funzione $f$ ammette un punto fisso.
}
\Hint{La funzione $x\mapsto d(x,f(x))$ ammette minimo.}

\Problem{Punti di discontinuità}{
Esiste una funzione $f:\R\to\R$ tale che sia continua in $x\in\R$ se e solo se $x\in\Q$?
}
\Hint{La risposta è no e per dimostrarlo si può passare dal fatto che i punti di discontinuità di $f$ sono un insieme magro (nell'ipotesi in cui su $\Q$ la funzione sia continua) e concludere con Baire.}

\Problem{Misura del tetraedro \texorpdfstring{$n$}{n}-dimensionale}{
Calcolare la misura del tetraedro $n$-dimensionale di lato unitario. Con tetraedro $n$-dimensionale si intende l'inviluppo convesso di $n+1$ punti in $\R^n$ equidistanti tra di loro.
}
\Hint{E se calcolassimo la misura in $\R^{n+1}$ al poso che in $\R^n$? Allora potremmo scegliere come vertici del tetraedro la base canonica...}

\Problem{Integrale di Eulero}{
Dimostrare l'uguaglianza
\begin{equation*}
	\int_{-\infty}^{\infty}e^{-x^2}\de x=\sqrt\pi
\end{equation*}
}
\Hint{Considerate l'identità
\begin{equation*}
	\left(\int_{-\infty}^{\infty}e^{-x^2}\right)^2 = \int_{\R^2}e^{-(x^2+y^2)}\de(x,y)\punto
\end{equation*}
Alternativamente, se conoscete la funzione gamma di Eulero, potreste sfruttare il fatto che, ad esempio con la formula di duplicazione, potete calcolare $\Gamma(\frac12)$.
}

\Problem{Quasi l'esponenziale}{
Fissato $k=10^{10000}$, mostrate la stima
\begin{equation*}
	\left\lvert\sum_{n=0}^k (-1)^n \frac{k^n}{n!^2}\right\lvert\le 1 \punto
\end{equation*}
Riuscite a migliorarla?
}
\Hint{Il comportamento asintotico delle soluzioni di equazioni differenziali può essere studiato con stime di tipo energia.}
\Solution{Sia $y:\R\to\R$ la funzione analitica definita come
\begin{equation*}
	y(x)=\sum_{n=0}^{\infty} (-1)^n\frac{x^n}{n!^2}
\end{equation*}
e sia $u:\R\to\R$ definita come $u(x)=x\cdot y'(x)$. 
È facile verificare che $u''=-\frac ux$ e che quindi la quantità $E(x)=\frac{u^2}x+{u'}^2$ ha come derivata $-\frac{u^2}{x^2}$.

A questo punto notiamo, che per $x\ge 1$, vale
\begin{equation*}
	y(x)^2=u'(x)^2\le E(x)\le E(1)=u(1)^2+{u'(1)}^2\le \left(\frac14\right)^2+\left(\frac12\right)^2<\frac12 \virgola
\end{equation*}
da cui la tesi discende facilmente.

Per avere stime migliore per $y$, si deve studiare come si comporta asintoticamente l'energia. Euristicamente sembra valere $E(x)\propto\frac1{\log^2(x)}$.
}

\Problem{Differenziale antisimmetrico}{
Se $f:\R^n\to\R^n$ è una funzione tale che il differenziale $\de f$ è antisimmetrico in ogni punto, allora $f$ è una funzione affine.
}
\Hint{Applicate il lemma di Schwartz (la commutazione delle derivate) alle componenti di $f$.}

\Problem{Serie alternante}{
Dati due numeri reali positivi $a,b$ con $b<1$, la serie $\sum_{n=1}^\infty \frac{\cos n^b}{n^a}$ converge se e solo se $a+b>1$.
}
\Hint{Quando $a+b\le 1$, al variare di $k\in\N$, considerate la somma parziale
\begin{equation*}
	\sum_{2k\pi\le n^b\le (2k+\frac14)\pi}\frac{\cos n^b}{n^a}\punto
\end{equation*}
Viceversa, per mostrare la convergenza, provate a riscrivere la serie come somma di Abel.
}