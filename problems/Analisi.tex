\section{Analisi}

\Problem{Contrazioni in un compatto}{
Sia $(X,d)$ uno spazio metrico compatto ed $f:X\to X$ una funzione tale che
\begin{equation*}
	\forall x,y\in X: d\left(f(x),f(y)\right)<d(x,y)\punto
\end{equation*}
Allora la funzione $f$ ammette un punto fisso.
}
\Hint{La funzione $x\mapsto d(x,f(x))$ ammette minimo.}

\Problem{Punti di discontinuità}{
Esiste una funzione $f:\R\to\R$ tale che sia continua in $x\in\R$ se e solo se $x\in\Q$?
}
\Hint{La risposta è no e per dimostrarlo si può passare dal fatto che i punti di discontinuità di $f$ sono un insieme magro (nell'ipotesi in cui su $\Q$ la funzione sia continua) e concludere con Baire.}

\Problem{Misura del tetraedro \texorpdfstring{$n$}{n}-dimensionale}{
Calcolare la misura del tetraedro $n$-dimensionale di lato unitario. Con tetraedro $n$-dimensionale si intende l'inviluppo convesso di $n+1$ punti in $\R^n$ equidistanti tra di loro.
}
\Hint{E se calcolassimo la misura in $\R^{n+1}$ al poso che in $\R^n$? Allora potremmo scegliere come vertici del tetraedro la base canonica...}

\Problem{Integrale di Eulero}{
Dimostrare l'uguaglianza
\begin{equation*}
	\int_{-\infty}^{\infty}e^{-x^2}\de x=\sqrt\pi
\end{equation*}
}
\Hint{Considerate l'identità
\begin{equation*}
	\left(\int_{-\infty}^{\infty}e^{-x^2}\right)^2 = \int_{\R^2}e^{-(x^2+y^2)}\de(x,y)\punto
\end{equation*}
}