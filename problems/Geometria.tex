\section{Geometria}

\Problem{Spazio euclideo senza alcuni punti}{
Fissato $n\ge 1$ e $h\not=k$ interi non negativi distinti, dimostrare che $\R^n$ senza $h$ punti non è omeomorfo a $\R^n$ senza $k$ punti.
}
\Hint{Considerate un'esaustione in compatti.}

\Problem{Gruppo fondamentale di una sfera con punti identificati}{
Sia $X$ lo spazio ottenuto identificando due punti di $S^2$. Chi è il gruppo fondamentale di $X$?
}

\Problem{Segmenti paralleli su una curva}{
Sia $\gamma:\cc{0}{1}\to\R^2$ una curva continua con $\gamma(0)=(0, 0)$ e $\gamma(1)=(1, 0)$.
Mostrare che, fissato $0<d<1$, esistono due tempi $t_1,t_2\in\cc{0}{1}$ tali che $\gamma(t_1)$ e $\gamma(t_2)$ formano un segmento parallelo all'asse delle $x$ e di lunghezza $d$ o $1-d$.
}
\Hint{È equivalente ad affermare che una curva sul cilindro omotopica al generatore del gruppo fondamentale interseca una rotazione di se stessa. A questo punto, dopo aver ``pulito'' $\gamma$, si ritorna opportunamente sul piano e si nota che qualcosa non torna per Jordan.}