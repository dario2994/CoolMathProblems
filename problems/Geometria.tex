\section{Geometria}

\Problem{Spazio euclideo senza alcuni punti}{
Fissato $n\ge 1$ e $h\not=k$ interi non negativi distinti, dimostrare che $\R^n$ senza $h$ punti non è omeomorfo a $\R^n$ senza $k$ punti.
}
\Hint{Considerate un'esaustione in compatti.}

\Problem{Gruppo fondamentale di una sfera con punti identificati}{
Sia $X$ lo spazio ottenuto identificando due punti di $S^2$. Chi è il gruppo fondamentale di $X$?
}

\Problem{Segmenti paralleli su una curva}{
Sia $\gamma:\cc{0}{1}\to\R^2$ una curva continua con $\gamma(0)=(0, 0)$ e $\gamma(1)=(1, 0)$.
Mostrare che, fissato $0<d<1$, esistono due tempi $t_1,t_2\in\cc{0}{1}$ tali che $\gamma(t_1)$ e $\gamma(t_2)$ formano un segmento parallelo all'asse delle $x$ e di lunghezza $d$ o $1-d$.
}
\Hint{È equivalente ad affermare che una curva sul cilindro omotopica al generatore del gruppo fondamentale interseca una rotazione di se stessa. A questo punto, dopo aver ``pulito'' $\gamma$, si ritorna opportunamente sul piano e si nota che qualcosa non torna per Jordan.}

\Problem{Coni convessi chiusi}{
	Dato un cono convesso chiuso $C\subseteq \R^n$ sono equivalenti le seguenti affermazioni:
	\begin{itemize}
		\item Una retta è contenuta in $C$.
		\item Per ogni funzionale $L\in(\R^n)'$ esiste $x\in C\setminus\{0\}$ tale che $Lx\ge 0$.
	\end{itemize}
}
\Hint{Considerate $C\cap S^{n-1}$.}
\Solution{Mostriamo che se $C$ non contiene rette allora esiste un funzionale che valutato su $C$ è ovunque negativo.
	
	L'insieme $D = C\cap S^{n-1}$ è chiuso e convesso e senza punti antipodali.
	
	Intersecando $D$ con $x_n=0$, per induzione, si trova che, senza perdita di generalità, $D$ è disgiunto dall'insieme $x_{n-1}\le 0$.
	
	Denotiamo ora con $S$ il sottospazio vettoriale di $\R^n$ dato da $x_n=x_{n-1}=0$ e con $P$ il piano ortogonale ad $S$. Definiamo ora la funzione angolo $\theta:P\to S^1$ che associa ad ogni elemento di $P$ l'angolo che forma con il vettore $e_n$. È ovvio che, a meno di proiettare su $P$, la funzione $\theta$ si estende a tutto $S^{n-1}$. % FIXME: Sarebbe da definire meglio l'angolo
	
	Chiamiamo $A$ l'immagine di $D$ tramite $\theta$. 
	Per quanto detto sopra, di certo $A$ non contiene $0$. Inoltre, visto che $D$ è convesso, anche $A$ è convesso.
	Allora, facilmente, si ha che $A$ è contenuto in una semicirconferenza e, prendendo la controimmagine della semicirconferenza tramite $\theta$, si trova che $D$ è contenuto in un semispazio.
	
	Rimane solo da verificare che, a meno di spostare di pochissimo questo semispazio, $D$ non è neanche sul bordo del semispazio e quindi esiste un funzionale che falsifica l'ipotesi. Lasciamo al lettore quest'ultimo passo.
}

\Problem{Allacciando colonne}{
	Re Mida ha un salone rotondo con $n\in\N$ colonne ed ha una corda d'oro di inestimabile valore (che potete assumere lunga a piacere). La corda è ``aperta'', nel senso che le estremità non sono collegate.
	
	Egli vuole allacciare (cioè disporre a piacimento e poi fondere le estremità) la corda intorno alle colonne in modo che non sia possibile rubarla. 
	Teme però il cedimento del salone e vorrebbe perciò che fosse possibile portare via la corda non appena una qualunque delle $n$ colonne crolli.
	Può re Mida riuscire in questo particolare allacciamento?
}
\Hint{Considerate i gruppi fondamentali (e le loro immersioni canoniche l'uno nell'altro) del piano euclideo senza $n$ o $n-1$ punti. Mostrate quindi per induzione che Re Mida può allacciare la corda come richiesto. }
\Solution{Denotiamo con $\pi(p_1, p_2, \dots, p_k)$ il gruppo fondamentale di $\R^2$ privato dei punti $p_1, p_2,\dots, p_k$ (tali punti rappresentano le colonne del salone). 
	
	È fatto noto che come gruppo $\pi(p_1, p_2, \dots p_k)$ è $\overbrace{\Z\ast \cdots \ast \Z}^{\text{$k$ volte}}$.
	Inoltre l'omomorfismo canonico indotto dall'immersione da $\pi(p_1, p_2, \dots p_k)$ in $\pi(p_1, p_2, \dots, \hat{p_i}, \dots, p_k)$ visto sui rappresentanti canonici dei gruppi liberi consiste nell'ignorare l'$i$-esimo rappresentante. Denotiamo con $\iota_i^k$ la mappa appena descritta tra i gruppi liberi
	\begin{equation}
		\iota_i^k:\overbrace{\Z\ast \cdots \ast \Z}^{\text{$k$ volte}} \to \overbrace{\Z\ast \cdots \ast \Z}^{\text{$k-1$ volte}} \punto
	\end{equation}
	Il problema è ora equivalente a chiedersi se esiste una parola $w\in\overbrace{\Z\ast \cdots \ast \Z}^{\text{$n$ volte}}$ diversa dall'identità tale che $\iota_i^n(w)$ è l'identità per ogni $1\le i\le n$.
	
	Mostriamo per induzione su $n$ che tale parola esiste. Per $n=1$ è banale. Se $w_{n-1}$ è una parola che rispetta per $n-1$, è facile convincersi che $x_nw_{n-1}x_n^{-1}w_{n-1}^{-1}$ funziona per $n$ (dove con $x_1,\dots, x_n$ stiamo denotando i rappresentanti canonici dei gruppi liberi).
}